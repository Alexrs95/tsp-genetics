\section{Implementation}

\subsection{Representation}
The origial code employed adjacency representation. EXPLAIN ADJACENCY REPRESENTATION AND EXAMPLE

In our implementation, we have deceided to use path representation. EXPLAIN PATH REPRESENTATION AND EXAMPLE. GIVE REASONS

\subsection{Mutation}
    - Insertion mutation ( http://mnemstudio.org/genetic\-algorithms\-mutation.htm )
 EXPLAIN WHAT MUTATION IS
 
 The mutation operator selected for this problem is insertion mutation. EXPLAIN HERE WHY AND WHAT IT IS


0 1 \textbf{2} 3 4 5 6 7

Take the 2 out of the sequence,

0 1 3 4 5 6 7

and reinsert the 2 at a randomly chosen position:

0 1 3 4 5 \textbf{2} 6 7

\subsection{Crossover}
  - Order Crossover ( http://www.dca.fee.unicamp.br/\~gomide/courses/EA072/artigos/Genetic\_Algorithm\_TSPR\_eview\_Larranaga\_1999.pdf )

  EXPLAIN WHAT CROSSOVER IS

  EXPLAIN ORDER CROSSOVER

  EXAMPLE

\subsection{Fitness Function}
The fitness functio has been changed 